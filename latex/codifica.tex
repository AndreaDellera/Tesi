\section{Codifica}
Le note, estratte da partiture scritte in formato MusicXML, devono avere una codifica per essere interpretate dalla rete. Le caratteristiche considerate sono cinque:
\begin{itemize}
\item[-]nome della nota;
\item[-]alterazione;
\item[-]durata;
\item[-]punto di valore;
\item[-]ottava.
\end{itemize}
Di queste cinque il nome e l'alterazione sono unite in un'unica codifica visto che sono strettamente correlate tra loro, cos\`i come la durata e punto di valore.

\subsection{Codifica delle note}
Le note nel sistema europeo sono chiamate \emph{la, si, do, re, mi, fa, sol}, che nel sistema americano corrispondono a \emph{A, B, C, D, E, F, G}; in questo scritto verr\`a utilizzato il secondo formato, perch\`e pi\`u compatto e sintetico.
Per codificare tutte le note servono almeno quattro bit, questo perch\`e oltre alle sette naturali riportate sopra abbiamo anche quelle alterate dai $\sharp$ (diesis) e dai $\flat$ (bemolle). Va ricordato per\`o che introducendo entrambe le alterazioni nella codifica si hanno note ridondati dal punto di vista sonoro. Infatti se G $\sharp$ e A $\flat$ indicano due note diverse, perch\`e cambia la tonalit\`a in cui vengono usate, il suono che viene prodotto quando sono suonate \`e per\`o lo stesso. Ecco perch\`e nella codifica utilizzeremo solo il diesis.

\begin{table}[ht]
\centering
\begin{tabular}{| l  c |}
\multicolumn {2}{c}{\textbf{Codifica delle note}}\\
\hline
A& 0000\\\hline
A$\sharp$& 0001\\\hline
B& 0010\\\hline
C& 0011\\\hline
C$\sharp$& 0100\\\hline
D& 0101\\\hline
D$\sharp$& 0110\\\hline
E& 0111\\\hline
F& 1000\\\hline
F$\sharp$& 1001\\\hline
G& 1010\\\hline
G$\sharp$& 1011\\\hline
\end{tabular}
\end{table}

\subsection{Codifica della durata}
Per la durata della nota il ragionamento \`e analogo. 	La durata massima di una nota \`e \setmetera{4}{4} (\Ganz). Si trovano tutte nella forma $\frac{1}{2^n}$ dove $0\leq n \leq +\infty$. Convenzionalmente per\`o le prime tre della sequenza sono descritte come \setmetera{4}{4} (\Ganz), \setmetera{2}{4} (\Halb), \setmetera{1}{4} (\Vier). Le durate che verranno codificate arriveranno fino ad \setmetera{1}{64} visto che velocit\`a pi\`u piccole di questa non vengono mai utilizzate.
Come fatto precedentemente verr\`a assegnata una sequenza di bit ad ogni durata.\\
\begin{table}[ht]
\renewcommand\arraystretch{1.4}
\centering
\begin{tabular}{| l  c |}
\multicolumn {2}{c}{\textbf{Codifica delle durate}}\\
\hline
\setmetera{4}{4}& 000\\\hline
\setmetera{2}{4}& 001\\\hline
\setmetera{1}{4}& 010\\\hline
\setmetera{1}{8}& 011\\\hline
\setmetera{1}{16}& 100\\\hline
\setmetera{1}{32}& 101\\\hline
\setmetera{1}{64}& 110\\\hline
\end{tabular}
\end{table}\\
Un'altra variabile che entra in gioco quando si parla di durata \`e il \emph{punto di valore} (\Pu). Questo strumento aumenta la durata della nota della sua met\`a e per codificarlo useremo un bit che sar\`a 0 se non c'\`e e 1 in caso contrario.

\subsection {Codifica dell'ottava}
In ci\`o che va a costituire una nota una parte importante \`e l'\emph{ottava}. L'ottava costituisce l'altezza della nota, un'offset rispetto a quella pi\`u bassa. Si pu\`o pensarla come una somma di $n * 12semitoni$ rispetto alla nota pi\`u bassa.
Anche qui la codifica che verr\`a seguita \`e binaria. Essendoci undici ottave, $C_{-1};...; C_9$, dovrebbero essere usati 4 bit per la rappresentazione ma, visto che le ottave $C_{-1}, C_0$ e $C_9$ non vengono praticamente mai utilizzate verranno impiegati tre bit per rappresentare le otto rimanenti.

\begin{table}[ht]
\centering
\begin{tabular}{| l  c |}
\multicolumn {2}{c}{\textbf{Codifica delle ottave}}\\
\hline
$C_1$& 000\\\hline
$C_2$& 001\\\hline
$C_3$& 010\\\hline
$C_4$& 011\\\hline
$C_5$& 100\\\hline
$C_6$& 101\\\hline
$C_7$& 110\\\hline
$C_8$& 111\\\hline
\end{tabular}
\end{table}

\subsection {Codifica delle pause}
Le pause sono uno strumento musicale molto comune, vengono utilizzate per dire ad uno strumento di non suonare. L'unico dato che portano \`e quello della durata, visto che non hanno una ottava di riferimento.
Verranno codificate con \emph{1111} nel campo del nome della nota e dell'ottava, in quanto rappresentano una nota inesistente.

\subsection{Esempi di codifica}
\begin{table}[ht]
\centering
\renewcommand\arraystretch{1.4}
\begin{tabular}{| c c |}
\hline
Do di \setmetera{4}{4}, della terza ottava:& 01100100000\\\hline
Do di \setmetera{1}{4}, della terza ottava:& 01100100100\\\hline
\end{tabular}
\end{table}

\subsection{Creazione degli esempi}
Per creare un esempio bisogna decidere quante note in input avr\`a la rete e, in base a quello, verr\`a costruito. Un esempio per una rete con cinque note in input \`e:\\
\begin{table}[ht]
\centering
\begin{tabular}{| c c c c c | c |}
\multicolumn {6}{c}{\textbf{Esempio}}\\
\hline
\begin{tabular}{ccc}
\multicolumn {3}{c}{$Nota1$}\\
\hline
ottava&nota&durata\\
\end{tabular}
&$Nota2$& $Nota3$& $Nota4$&$Nota5$&$Target$\\\hline
\end{tabular}
\end{table}