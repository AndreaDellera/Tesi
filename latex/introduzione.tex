\section{Introduzione}
% Il mondo della musica \`e sempre stato qualcosa di astratto, di impalpabile. Regge ormai da decenni la figura del rocker che, grazie alla sua chitarra, anima folle sempre pi\`u grandi e si oppone alle scelte del governo corrente. Un po' pi\`u in disuso \`e la figura del musicista in voga negli anni d'oro della musica classica, da Mozart lo scrittore maltrattato dalla vita a Beethoven che, nonostante la sordit\`a, riusc\`i a comporre un capolavoro come la \emph{sonata al chiaro di luna}.
%Il processo creativo che sta dietro alla stesura di un brano pu\`o essere riprodotto da un'entit\`a come un computer? Pu\`o un algoritmo, una tecnica di programmazione, portare una macchina a produrre delle melodie?
Lo scopo di questa tesi \`e di creare delle melodie utilizzando tecniche di machine learning per portare il computer ad imparare brani gi\`a scritti per poi creare delle melodie originali. Verr\`a perseguito facendo inoltre un confronto tra due tipologie di reti neurali: Feed Forward Netwoks (FFN) e Recurrent Neural Networks (RNN).\\
I passi seguiti per giungere allo scopo sono:
\begin{itemize}
\item[-]la definizione delle reti, con tutte le variabili del caso da definire come il numero di input, il numero di neuroni nello strato nascosto ed il numero di output;
\item[-]la definizione di una codifica per le note, con la scelta delle propriet\`a da tenere in considerazione come l'ottava, la durata ed il nome della nota;
\item[-]la scelta di un algoritmo per l'allenamento delle reti e delle relative variabili, come ad esempio il tipo di apprendimento e la gestione del dataset;
\item[-]l'apprendimento di sequenze di note particolari, messe a forma di cerchio o di otto;
\item[-]l'apprendimento di intere canzoni.
\end{itemize}
Lavori di questo tipo sono gi\`a stati fatti, con risultati pi\`u e meno soddisfacenti, e si rimanda a~\cite{todd1989} per una versione pi\`u semplice del problema mentre si consiglia\\\url{http://www.hexahedria.com/2015/08/03/composing-music-with-recurrent-neural-networks/} per una versione pi\`u approfondita.

