\section{Introduzione}
Lo scopo di questa tesi \`e di creare delle melodie utilizzando tecniche di machine learning per portare il computer ad imparare brani gi\`a scritti per poi creare delle melodie originali. Verr\`a perseguito facendo inoltre un confronto tra due tipologie di reti neurali: Feed Forward Netwoks (FFN) e Recurrent Neural Networks (RNN).\\
I passi seguiti per giungere allo scopo sono:
\begin{itemize}
\item[-]la definizione delle reti, con l'architettura da utilizzare e le variabili, come il numero di input, il numero di neuroni nello strato nascosto ed il numero di output;
\item[-]la definizione di una codifica per le note, con la scelta delle propriet\`a da tenere in considerazione come l'ottava, la durata ed il nome della nota;
\item[-]la scelta di un algoritmo per l'allenamento delle reti e delle relative variabili, come ad esempio il tipo di apprendimento e la gestione del dataset;
\item[-]l'apprendimento di sequenze di note particolari, disposte in modo da formare delle figure come il cerchio o l'otto;
\item[-]l'apprendimento di intere canzoni.
\end{itemize}
Lavori di questo tipo sono gi\`a stati fatti, con risultati pi\`u e meno soddisfacenti, e si rimanda a~\cite{todd1989} per una versione pi\`u semplice del problema mentre si consiglia~\cite{JohnsonRNN} per una versione pi\`u approfondita.

