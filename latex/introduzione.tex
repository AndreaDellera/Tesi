\section{Introduzione}
Il mondo della musica \`e sempre stato qualcosa di astratto, di impalpabile. Regge ormai da decenni la figura del rocker che, grazie alla sua chitarra, anima folle sempre pi\`u grandi e si oppone alle scelte del governo corrente. Un po' pi\`u in disuso \`e la figura del musicista in voga negli anni d'oro della musica classica, da Mozart lo scrittore maltrattato dalla vita a Beethoven che, nonostante la sordit\`a, riusc\`i a comporre un capolavoro come la \emph{sonata al chiaro di luna}.
Ma il processo creativo che sta dietro alla stesura di un brano pu\`o essere riprodotto da un qualcosa che non pensa come un computer? Pu\`o un algoritmo, una tecnica di programmazione, portare una macchina a produrre delle melodie?
Lo scopo di questa tesi \`e di creare delle melodie utilizzando il machine learning per potare la macchina ad acquisire, ad imparare, brani gi\`a scritti per poi creare delle melodie originali e verr\`a perseguito facendo un confronto tra due tipologie di reti neurali: Feed Forward Netwoks (FFN) e Recurrent Neural Network (RNN).\\
I passi seguiti sono la definizione di una codifica per le note, la definizione delle reti e l'apprendimento di diverse sequenze di note, a partire da scale fino ad intere canzoni.
