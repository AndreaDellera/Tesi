\section{Conclusioni}
A  lavoro finito fatto si \`e capito che le Feed Forward Networks possano imparare semplici sequenze di note come scale o arpeggi ma non riescono a dare un significato a ci\`o che generano, a creare della musica. Le Recurrent Neural Networks invece hanno la capacit\`a di generare note in modo pi\`u complesso e vario tenendo conto della storia passata ma, arrivati ad un certo numero di note prodotte, perdono il contesto creando s\`i note che si possono definire giuste in quanto appartenenti ad una certa tonalit\`a o di una certa durata, ma che all'orecchio umano risultano scollegate l'una dall'altra. Quindi neanche a questo tipo di rete si pu\`o chiedere di generare musica in modo completamente esatto.\\
Inoltre tutto dipende dal numero di note che si decide di dare in input alla rete, dato che con poche note non si riescono a predire bene i target, mentre con troppe note crescono considerevolmente i tempi per l'allenamento della rete. Oltre a questo le reti, quando generano delle note, non riescono a tenere conto del tempo in cui sono stati scritti i brani usati per allenarle (la metrica delle canzoni).\\
Per approfondire ulteriormente l'argomento sarebbe necessario implementare delle reti che riescano ad imparare da vari brani senza presentare gli errori ed i problemi spiegati precedentemente, cosa che in questa tesi per mancanza di tempo e per l'elevata complessit\`a non si \`e riusciti a fare.