\section{Conclusioni}
A conclusione del lavoro fatto si \`e giunti alla conclusione che le Feed Forward Networks possano imparare semplici sequenze di note come scale o arpeggi ma non riescano a dare un significato a ci\`o che generano, a creare della musica. Le Recurrent Neural Networks invece hanno la capacit\`a di generare note in modo pi\`u complesso e vario tenendo conto della storia passata ma, arrivati ad un certo numero di note, perdono il contesto creando si note che si possono definire giuste in quanto appartenenti ad una certa tonalit\`a o di una certa durata ma che all'orecchio umano risultano scollegate l'una dall'altra. Quindi neanche a questo tipo di rete si pu\`o chiedere di generare musica in modo completamente esatto.\\
Oltre a questo le reti quando generano delle note non riescono a tenere conto del tempo in cui \`e scritto il brano.
\section{Lavori futuri}
Questo progetto \`e stato sviluppato come lavoro di tesi e tirocinio. Tuttavia verr\`a portato avanti, implementando le reti utilizzando un proprio codice e non quello di una libreria esterna ed il goal del progetto sar\`a creare diverse reti che creino musica di diversi generi.