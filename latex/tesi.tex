\documentclass[a4paper]{article}
\usepackage{cite}
\usepackage{harmony}
\usepackage{amsmath}
\newcommand{\setmetera}[2]{\ensuremath{\genfrac{}{}{0pt}{}{#1}{#2}}} % \setmetera{4}{4} per scrivere il tempo 4/4


\begin{document}

\begin{titlepage}
\newcommand{\HRule}{\rule{\linewidth}{0.5mm}} % Defines a new command for the horizontal lines, change thickness here


\textsc{\LARGE Universit\`a di Trento}\\[0.8cm] % Name of your university/college
\textsc{\Large Corso di laurea in informatica}\\[0.8cm] % Major heading such as course name

\HRule\\[0.8cm]
{\huge \bfseries A Musical Version of Recurrent Neural Network}\\[0.4cm] % Title of your document
\HRule\\[2cm]

\begin{minipage}{0.4\textwidth}
\begin{flushleft} \large
\begin{tabular}{ll}
Dellera \textsc{Andrea} & \makebox[2cm][r]{158365} \\
\end{tabular}
\end{flushleft}
\end{minipage}

\vfill % Fill the rest of the page with whitespace
{\large \today}\\[3cm] % Date, change the \today to a set date if you want to be precise
\end{titlepage}

\newpage

\section{Introduzione}
Il mondo della musica � sempre stato qualcosa di astratto, di impalpabile. Regge ormai da decenni la figura del rocker che, grazie alla sua chitarra, anima folle sempre pi� grandi e si oppone alle scelte del governo corrente. Un po' pi\`u in disuso \`e la figura del musicista in voga negli anni d'oro della musica classica, da Mozart lo scrittore maltrattato dalla vita a Beethoven che, nonostante la sordit\`a, riusc\`i a comporre un capolavoro come la \emph{sonata al chiaro di luna}.\\
Ma il processo creativo che sta dietro alla stesura di un brano pu\`o essere riprodotto da un qualcosa che non pensa come un computer? Pu\`o un algoritmo, una tecnica di programmazione, portare una macchina a produrre delle melodie?\\
Lo scopo di questa tesi \`e di creare delle melodie utilizzando una tecnica di machine learning per potare la macchina ad acquisire brani gi\`a scritti e successivamente creare delle melodie.\\

\newpage
\section{Codifica}
Il primo passo per l'implementazione di questo obiettivo \`e la codifica delle note. Qui entrano in gioco vari parametri (sono stati presi in considerazione i pi� essenziali): 
\begin{itemize}
\item[-] nome della nota;
\item[-] durata della nota;
\item[-] presenza del punto di valore.
\end{itemize}
\subsection{Codifica delle note}
Le note nel sistema europeo sono chiamate \emph{la, si, do, re, mi, fa, sol} che nel sistema europeo corrispondono a \emph{A, B, C, D, E, F, G}; in questo paper verr\`a utilizzato il secondo, perch\`e pi\`u compatto e sintetico.\\
Come era gi\`a stato analizzato da Todd ~\cite{todd1989} il problema principale consiste nel rendere pi� generale possibile una certa melodia, rendendola quindi scollegata dalla chiave in cui \`e stata scritta\footnote{esistono diverse tonalit� in cui � possibile scrivere una certa melodia, determinate dalla successione delle note (TODO)}.
Il metodo migliore per perseguire questo obiettivo \`e rappresentare la distanza tra due note in semitoni, cio\`e la differenza pi\`u piccola tra due note. Ecco che quindi tutto ci\`o di cui si avr\`a bisogno sar\`a la nota di partenza. Ecco quindi che la stessa melodia, diversa a seconda della chiave di lettura, diventa uguale quando si guarda la differenza tra le note.\\
\begin{table}[h]
\centering
\begin{tabular}{| l | c c c c c c c |}
\multicolumn{8}{c}{\textbf{Notazioni}}\\
\hline
Americana:& A& B& C& D& E& F& G\\
\hline
Europea:& La& Si& Do& Re& Mi& Fa& Sol\\
\hline
\end{tabular}
\end{table}

\begin{table}[h]
\centering
\begin{tabular}{| c c c c |}
\multicolumn{4}{c}{\textbf{Intervalli}}\\
\hline
 A& C& E& G\\
 A& +3& +4& +3\\
\hline
 C& E& G& Bb\\
 C& +3& +4& +3\\
 \hline
 \end{tabular}
 \end{table}
 
Per codificare tutte le note servono almeno quattro bit, questo perch� oltre alle sette naturali riportate sopra abbiamo anche quelle alterate dai $\sharp$ e dai $\flat$. Va ricordato per\`o che introducendo entrambe le alterazioni nella codifica si hanno note ridondati dal punto di vista sonoro. Infatti se G $\sharp$ e A $\flat$ indicano due note diverse, perch\`e cambia la tonalit� in cui vengono usate, il suono che viene prodotto quando sono suonate \`e per\`o lo stesso. Ecco perch� nella codifica utilizzeremo solo il diesis.\\
\begin{table}[h]
\centering
\begin{tabular}{| l  c |}
\multicolumn {2}{c}{\textbf{Codifica}}\\
\hline
A& 0000\\\hline
A$\sharp$& 0001\\\hline
B& 0010\\\hline
C& 0011\\\hline
C$\sharp$& 0100\\\hline
D& 0101\\\hline
D$\sharp$& 0110\\\hline
E& 0111\\\hline
F& 1000\\\hline
F$\sharp$& 1001\\\hline
G& 1010\\\hline
G$\sharp$& 1011\\\hline
\end{tabular}
\end{table}

Con questa codifica il sistema ad intervalli no viene perso. Per, date due note $x_1$ e $x_2$, se $x_1 \leq x_2$ l'intervallo si otterr� facendo $x_2 - x_1$ mentre se $x_2 \leq x_1$ si far\`a $x_1- x_2$.\\ 
\\
\subsection{Codifica della durata}
Per la durata della nota il ragionamento \`e analogo. 	La durata massima di una nota \`e \setmetera{4}{4} (\Ganz). Si trovano tutte nella forma $\frac{1}{2^n}$ dove $0\leq n \leq +\infty$. Convenzionalmente per\`o le prime tre della sequenza sono descritte come \setmetera{4}{4} (\Ganz), \setmetera{2}{4} (\Halb), \setmetera{1}{4} (\Vier).
Le durate che verranno codificate arriveranno fino ad \setmetera{1}{64}.\\
Come fatto precedentemente verr\`a assegnata una sequenza di bit ad ogni durata.\\
\begin{table}[h]
\renewcommand\arraystretch{1.4}
\centering
\begin{tabular}{| l  c |}
\multicolumn {2}{c}{\textbf{Codifica}}\\
\hline
\setmetera{4}{4}& 000\\\hline
\setmetera{2}{4}& 001\\\hline
\setmetera{1}{4}& 010\\\hline
\setmetera{1}{8}& 011\\\hline
\setmetera{1}{16}& 100\\\hline
\setmetera{1}{32}& 101\\\hline
\setmetera{1}{64}& 110\\\hline
\end{tabular}
\end{table}\\
Un'altra variabile che entra quando si parla di durata \`e il \emph{punto di valore} (\Pu). Questo strumento aumenta la durata della nota della sua met\`a e per codificarlo useremo un bit che sar� 0 se non c'\`e e 1 in caso contrario.\\

\subsection {Codifica dell'ottava}
In ci� che va a costituire una note una parte importante \`e l'\emph{ottava}. L'ottava costituisce l'altezza della nota, un'offset rispetto a quella pi\`u bassa. Si pu\`o pensarla come una somma di $n * 12semitoni$ rispetto alla nota pi� bassa.
Anche qui la codifica che verr\`a seguita � binaria. Essendoci undici ottave, $C_{-1};...; C_9$, dovrebbero essere usati 4 bit per la rappresentazione ma, visto che le ottave $C_{-1}, C_0$ e $C_9$ non vengono praticamente mai usate ne verranno usati 3 per rappresentare le otto rimanenti.

\begin{table}[h]
\centering
\begin{tabular}{| l  c |}
\multicolumn {2}{c}{\textbf{Codifica}}\\
\hline
$C_1$& 000\\\hline
$C_2$& 001\\\hline
$C_3$& 010\\\hline
$C_4$& 011\\\hline
$C_5$& 100\\\hline
$C_6$& 101\\\hline
$C_7$& 110\\\hline
$C_8$& 111\\\hline
\end{tabular}
\end{table}

\subsection {Codifca delle pause}
Le pause sono uno strumento musicale molto comune, vengono utilizzate per dire ad uno strumento di non suonare. L'unico dato che portano \`e quello della durata, visto che non hanno una ottava di riferimento.
Verranno codificate con \emph{1111} nel campo di codifica della nota, in quanto denotano una nota inesistente

\subsection{Esempi di codifica}
\begin{table}[h]
\renewcommand\arraystretch{1.4}
\begin{tabular}{| c c |}
Do di \setmetera{4}{4}, della terza ottava:& 01100100000\\
Do di \setmetera{1}{4}, della terza ottava:& 01100100100\\
\end{tabular}
\end{table}

\newpage
\bibliographystyle{unsrt}
\bibliography{tesi}
\end{document}