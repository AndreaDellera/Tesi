Quando si \`e trattato di far imparare alle reti pi\`u di una canzone sono sorti alcuni problemi perch\`e queste in qualche modo dimenticavano le sequenze gi\`a imparate a favore delle ultime viste; inoltre si riscontravano problemi a livello del codice utilizzato per implementare le reti~\cite{schaul2010pybrain} che non si \`e riusciti ad identificare ma evidentemente presenti visto l'andamento degli errori di train e validazione.\\
Come si pu\`o vedere dai primi grafici (figure~\ref{fig:rrn_errors_all} e~\ref{fig:ffn_errors_all}) l'andamento degli errori non ha il comportamento che ci si aspetta da una rete. Se si valutano gli errori medi si evince un andamento pi\`u classico ma guardando quelli per ogni sessione si vede che non \`e normale e non c\`e stato abbastanza tempo per trovare un motivo a questo comportamento.\\
Una possibile spiegazione pu\`o essere che 
\begin{figure}[!htb]
		\begin{minipage}[b]{8.5cm}
		\centering
		\includegraphics[width=1\textwidth]{img/rnn_songs_all_errors.png}
		\caption{Errori di allenamento e validazione per pi\`u canzoni imparate dalla rete utilizzando una RNN.}
		\label{fig:rrn_errors_all}
	\end{minipage}
	\hspace{2mm} \hspace{3mm} \
	\begin{minipage}[b]{8.5cm}
		\centering
		\includegraphics[width=1\textwidth]{img/rnn_songs_kfold_errors.png}
		\caption{Errori medi di allenamento e validazione per pi\`u canzoni imparate dalla rete utilizzando una RNN.}
		\label{fig:rnn_errors_kfold}
	\end{minipage}
\end{figure}
\begin{figure}[!htb]
		\begin{minipage}[b]{8.5cm}
		\centering
		\includegraphics[width=1\textwidth]{img/ffn_songs_all_errors.png}
		\caption{Errori di allenamento e validazione per pi\`u canzoni imparate dalla rete utilizzando una FFN.}
		\label{fig:ffn_errors_all}
	\end{minipage}
	\hspace{2mm} \hspace{3mm} \
	\begin{minipage}[b]{8.5cm}
		\centering
		\includegraphics[width=1\textwidth]{img/ffn_songs_kfold_errors.png}
		\caption{Errori medi di allenamento e validazione per pi\`u canzoni imparate dalla rete utilizzando una FFN.}
		\label{fig:ffn_errors_kfold}
	\end{minipage}
\end{figure}

